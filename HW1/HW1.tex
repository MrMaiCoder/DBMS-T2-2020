\documentclass[a4paper, 11pt]{article}
\usepackage{fancyvrb, lipsum}
\usepackage{amsmath}
\usepackage{amssymb}
\usepackage{fancyhdr}
\usepackage{graphicx}


\usepackage[margin=1in]{geometry}
\usepackage{listings}
\usepackage{color}

\definecolor{dkgreen}{rgb}{0,0.6,0}
\definecolor{gray}{rgb}{0.5,0.5,0.5}
\definecolor{mauve}{rgb}{0.58,0,0.82}

\lstset{frame=tb,
	language=Java,
	aboveskip=3mm,
	belowskip=3mm,
	showstringspaces=false,
	columns=flexible,
	basicstyle={\small\ttfamily},
	numbers=none,
	numberstyle=\tiny\color{gray},
	keywordstyle=\color{blue},
	commentstyle=\color{dkgreen},
	stringstyle=\color{mauve},
	breaklines=true,
	breakatwhitespace=true,
	tabsize=3
}
%\begin{lstlisting}
%\end{lstlisting} write codes

%begin{algorithm}

\newcommand{\question}[2] {\vspace{.25in} \hrule\vspace{0.5em}
	\noindent{\bf #1: #2} \vspace{0.5em}
	\hrule \vspace{.10in}}
\renewcommand{\part}[1] {\vspace{.10in} {\bf (#1)}}

\newcommand{\myname}{Phairat Lin}
\newcommand{\myhwnum}{1}

\setlength{\parindent}{0pt}
\setlength{\parskip}{5pt plus 1pt}

\pagestyle{fancyplain}

\begin{document}
	
	\medskip                        % Skip a "medium" amount of space
	% (latex determines what medium is)
	% Also try: \bigskip, \littleskip
	
	\thispagestyle{plain}
	\begin{center}                  % Center the following lines
		{\Large ICCS240: Assignment \myhwnum} \\
		\myname \\
	\end{center}
	
	\question{Problem 1} 
	
	1. $2^{mn}$ \\
	2. $mn$

	\question{Problem 2}
	
	\part{1} $\Pi_{B}(R \bowtie S) =\Pi_{B}(R) \cap \Pi_{B}(S)$ \\
	For the left side, from the definition of natural join, \\
	$$\Pi_{B}(R \bowtie S) = \Pi_{R \cup S} (\sigma_{R.B = S.B} (R \times S)) = \{B | B \in R \cap S\}$$
	For the right side, 
	$$\Pi_{B}(R) \cap \Pi_{B}(S) = \{B | B \in R \cap S\} $$ 
	Hence, both sides are equivalent. $\square$
	
    \part{2} $\Pi_{A,C}(R \bowtie \sigma_{B=0}(S)) =\Pi_{A}(\sigma_{B=0}(R)) \times  \Pi_{C}(\sigma_{B=0}(S))$ \\
    $$ \Pi_{A,C}(R \bowtie \sigma_{B=0}(S)) =\Pi_{A, C}(\sigma_{B=0}(R) \times  \sigma_{B=0}(S))$$
    $$ \Pi_{A,C} \Pi_{A \cup C}(\sigma_{B=0}(R \times S)) = \Pi_{A,C} (\sigma_{B=0}(R)) \times  \sigma_{B=0}(S)$$
    $$ \sigma_{B=0}(R \times S) = \sigma_{B=0}(R) \times  \sigma_{B=0}(S)$$
    $$ \sigma_{B=0}(R \times S) = \sigma_{B=0} (R \times S)$$
    
    Hence, both sides are equivalent. $\square$
	
	\question{Problem 3}
	\part{1} 
	\vfill
	\begin{center} 
	\includegraphics[scale=0.6]{[DBMS]HW1.png} 
	\end{center}
	\vfill
	\part{2} \\
	propcompany(\underline{propname}: $string$, propno: $int$ ) \\
	apartment(\underline{aptaddress}: $string$, \underline{office}: $string$, units: $int$) \\
	parking(\underline{parkno}: $int$, status: $string$) \\
	utilcompany(\underline{utilno}: $int$, utilname: $string$, utiladdress: $string$) \\
	\question{Problem 4}
	
	\includegraphics[width=\textwidth]{[DBMS]HW1-2.png}
	\question{Problem 5}
	
	\begin{verbatim}
	beer(brand, standard_price, alcohol_percentage, country_brewed, country_sold)
	company(brand, HQ_location, year_founded)
	bar(name, location, brand_of_beer_sold, price_sold) 
	sale(bar, brand_of_beer, year_record, number_of_sold)
	\end{verbatim}
	
	\part{1} \\
	beer: \\
	PRIMARY KEY (brand) \\
	
	company: \\
	PRIMARY KEY (HQ\_location) \\
	FOREIGN KEY (brand) REFERENCES beer (brand) \\
	
	bar: \\
	PRIMARY KEY (name) \\
	FOREIGN KEY (brand\_of\_beer\_sold) REFERENCES beer (brand) \\
	
	sale: \\
	PRIMARY KEY: NONE \\
	FOREIGN KEY (bar) REFERENCES bar (name) \\
	FOREIGN KEY (brand\_of\_beer) REFERENCES beer (brand) \\
	
	\part{2a} 
	
	\begin{verbatim}
	SELECT name FROM beer WHERE country_brewed <> country_sold
	\end{verbatim}
	
	$\Pi_{name} \sigma_{country\_brewed != country\_sold} (beer)$
	
	\part{2b} 
	
	\begin{verbatim}
	SELECT SUM(number_of_sold) FROM sale GROUP BY year_record
	\end{verbatim}
	
	$\Pi_{SUM(number\_of\_sold)} \sigma\gamma_{year\_record}(sale)$
	
	\part{2c}
	
	\begin{verbatim}
	SELECT name, brand FROM bar, beer WHERE price_sold > standard_price
	\end{verbatim}
	
	$\Pi_{name, brand} \sigma_{price\_sold > standard\_price} (bar \times beer)$
	
	\question{Problem 6}
	
	\begin{verbatim}
	computer (maker, model, type, price)
	pc (model, speed, ram, storage)
	laptop (model, speed, ram, storage, screen)
	\end{verbatim}
	\part{1}
	\begin{verbatim}
	SELECT DISTINCT COUNT(maker) FROM computer GROUP BY type
	\end{verbatim}
	\part{2}
	\begin{verbatim}
	SELECT maker FROM computer ORDER BY ( 
	SELECT COUNT(model) FROM computer GROUP BY maker) 
	DESC WHERE rownum = 1
	\end{verbatim}
	\part{3}
	\begin{verbatim}
	SELECT compc.model, comlaptop.model, ABS(compc.price - comlaptop.price) difference
	FROM computer compc
	INNER JOIN computer comlaptop ON
	    compc.maker = comlaptop.maker
	WHERE compc.type = "pc"
	AND comlaptop.type = "laptop"
	AND ABS(compc.price - comlaptop.price) < 100
	\end{verbatim}

\end{document}